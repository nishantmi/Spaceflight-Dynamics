\documentclass[12pt, letterpaper]{article}
\usepackage[utf8]{inputenc}

\title{\textbf{Spaceflight Dynamics}\\ 
 		\large{POA	SOS $2020$}}
\author{Nishant Mittal - $190070038$}

\begin{document}

\maketitle

\noindent
\textbf{Timeline:}

\noindent
Week 1 : Brush up on basic kinematic equations and dive deeper into the initial maths (change of origin, rotation and translation in various coordinate systems, quaternion rotation) required to set foundation for further mathematical analysis. Along with that, learn more about how to use LaTex.\\

\noindent
Week 2: Look into the base physics required - particle dynamics, rigid body dynamics, two body problem.\\

\noindent
Week 3: Learn about aerodynamic aspects of rockets, the factors affecting them both in Earth atmosphere as well as in space. The underlying physics and corresponding mathematical treatment of the same.\\

\noindent
Week 4: Learn more about the space environment - radiation effects, maintaining temperature, meteors, meteorites, magnetic mirrors etc.\\

\noindent
Week 5: Learning about the propulsion techniques employed, launch sequences, the problems faced and how they are overcome. The multistage and single stage rocket concepts. Difference and benefits of bell thrusters and aerospikes.\\

\noindent
Week 6: A brief dive into orbital mechanics, orbital transfer manoeuvres and payload deployment. This will be used to further look into planetary fly-bys, gravitational turn trajectories, and interplanetary trajectories.\\

\noindent
Week 7: Learn about re-entry dynamics. plotting re-entry trajectories( topics like polynomial descent gradient), the physics and maths that go behind re-entries. \\

\noindent
Week 8: Analysis of the Voyager missions and their trajectories that have taken them to interstellar space. Analysis of SpaceX’s booster landings for more recent developments and to gain better understanding.\\

\noindent
\textbf{Sources:}

\noindent
Main Reference Source: William E. Wiesel, ‘Spaceflight Dynamics’, 2nd Ed., McGraw-Hill
Additional Material: Francis J.Hale, ‘Introduction to Space Flight’, Prentice Hall, 1994.
Apart from that Google and Youtube are always an open option to understand and look up more detail.

\end{document}